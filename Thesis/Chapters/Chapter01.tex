\section{Il problema}
Il problema del bin packing è così formulato:
\begin{quote}
Dati degli oggetti di volume differente, questi devono essere allocati in un numero finito di contenitori (Bin), ciascuno di una determinata capacità fissa, in modo da ridurre al minimo la quantità di Bin utilizzata.
\end{quote}
Un altro modo di formulare il problema è attraverso la programmazione lineare intera:
\begin{quote}	
	\begin{equation*}
		\begin{array}{ll@{}ll}
			\text{min}  & \displaystyle\sum\limits_{i=1}^{n} & y_i \\
			\text{s.t.} & \displaystyle\sum\limits_{j=1}^{n} & w_j x_{ij} \leq c y_i,  & i=1 ,..., n\\
		                & \displaystyle\sum\limits_{i=1}^{n} & x_{ij} = 1 & j=1 ,..., n\\
		             	& 								     & y_{j} \in \{0,1\}\\
		                & 								     & x_{ij} \in \{0,1\}\\
		\end{array}
\end{equation*}
	
\end{quote}
Questo problema è chiaramente di natura combinatorica, e in particolare appartiene all'insieme dei problemi NP-hard, mentre la versione di decisione del problema fa parte dell'insieme dei problemi NP-complete.