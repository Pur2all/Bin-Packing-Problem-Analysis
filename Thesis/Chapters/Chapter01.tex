\section{Il problema}
Spesso ci si può ritrovare a dover risolvere problemi in cui si devono disporre degli oggetti in contenitori al fine
di utilizzarne il minor numero possibile, ad esempio se si deve prendere un aereo bisogna organizzare tutte le proprie cose in delle
valigie da portare con sé, valigie che ovviamente avranno un costo di trasporto, di conseguenza si è interessati ad
utilizzare il minimo numero di valigie al fine di spendere il meno possibile per i bagagli. Questa è proprio una
classica applicazione del problema del bin packing, un problema che è più frequente di quanto si immagini.

\subsection{Formulazione}
Il problema del bin packing è così formulato:
\begin{quote}
	Dati degli oggetti di volume differente, questi devono essere allocati in un numero finito di
	contenitori (bin), ciascuno di una determinata capacità fissa, in modo da ridurre al minimo la
	quantità di contenitori utilizzata.
\end{quote}
\noindent
Un altro modo di formulare il problema è attraverso la programmazione lineare intera:
\begin{quote}	
	\begin{equation*}
		\begin{array}{ll@{}ll}
			\text{min}  & \displaystyle\sum\limits_{i=1}^{n} & y_i 					  & \\
			\text{s.t.} & \displaystyle\sum\limits_{j=1}^{n} & w_j x_{ij} \leq c y_i  & \forall i \in \{1, ..., n\}    \\
		                & \displaystyle\sum\limits_{i=1}^{n} & x_{ij} = 1 			  & \forall j \in \{1, ..., n\}    \\
		             	& 								     & y_{i} \in \{0,1\}	  & \forall i \in \{1, ..., n\}    \\
		                & 								     & x_{ij} \in \{0,1\}  	  & \forall i, j \in \{1, ..., n\} \\
		\end{array}
	\end{equation*}
	\begin{equation*}
		\begin{array}{ll@{}ll}
	    	\text{dove} &									& \\
	    				& w_j = \text{peso dell'oggetto } j	\text{, } w_j \in \mathbb{Z}^+ \text{, } w_j \leq c \text{ } \forall j \in \{1, ..., n\}  & \\ \\ 
		    			& c = \text{capacità di ogni bin, } c \in \mathbb{N} & \\ \\
		    			& y_i = 
		    				\begin{cases}
    							1 & \text{se il bin } i \text{ è utilizzato} \\
    							0 & \text{altrimenti}
							\end{cases}					    & \\ \\
						& x_{ij} = 
							\begin{cases}
    							1 & \text{se l'oggetto } j \text{ è assegnato al bin } i \\
    							0 & \text{altrimenti}
							\end{cases}					    & \\
		\end{array}
	\end{equation*}
\end{quote}
\noindent
In modo da rendere più chiara la comprensione viene fornito un esempio:
\begin{quote}
	Sia $ S $ un insieme di cardinalità $ n $ contenente i vari oggetti, indicati da un numero intero positivo, e $ s(i) $
	il volume dei vari oggetti $ \forall i \in S $ e sia $ c $ la capacità di un generico bin. \\
	Consideriamo la seguente istanza:
	\begin{equation*}
		\begin{array}{c}
			S = \{1, ..., 10\} \text{, quindi } |S| = n = 10 \text{, } c = 15 \\
	    \end{array}
	\end{equation*}
	\begin{equation*}
	    \begin{array}{cc}
			s(1) = 15	&	s(6) = 7   \\
			s(2) = 3	&	s(7) = 13  \\
			s(3) = 9	&	s(8) = 1   \\
			s(4) = 5	&	s(9) = 2   \\
			s(5) = 3	&	s(10) = 11 \\
		\end{array}
	\end{equation*}
	Una soluzione ottima per l'istanza è data dai bins:
	\begin{align*}
		B_1 = \{1\} ,\quad B_2 = \{7, 9\} ,\quad B_3 = \{4, 5, 6\} ,\quad B_4 = \{2, 8, 10\} ,\quad B_5 = \{3\}
	\end{align*}
	E quindi si può concludere che il numero minimo di bins da utilizzare per quest'istanza è $ 5 $.
\end{quote}

\subsection{Difficoltà}
Questo problema è chiaramente di natura combinatoria, e in particolare appartiene all'insieme dei problemi NP-hard, mentre
la versione di decisione del problema fa parte dell'insieme dei problemi NP-complete ed è così definita \cite{CaI_NP}:
\begin{quote}
	Istanza: Insieme finito $ U $ di oggetti, una taglia $ s(u) \in \mathbb{Z}^+ \: \forall u \in U $, una capacità intera
	per i bins $ B $ e un intero positivo $ K $. \\
	Domanda: Esiste una partizione di $ U $ in insiemi disgiunti $ U_1, U_2, ..., U_k $ tale che la somma delle taglie degli
	oggetti in ogni $ U_i $ è $ \leq B $.
\end{quote}
Nello specifico è un problema strongly NP-complete e questo può essere provato tramite una riduzione polinomiale da un altro
problema strongly NP-complete, ovvero il 3-partition. \\
Nonostante ciò comunque è possibile risolvere il problema del bin packing in tempo pseudo-polinomiale per ogni numero di
bins fissato $ \geq 2 $ \cite{BinPackingFixedSize}, e in tempo polinomiale per
ogni fissata capacità $ B $ enumerando tutte le possibili $ k^{p(1) + p(2) + ... + p(B)} $ assegnazioni degli oggetti ai
bins e trovare la più conveniente tramite ricerca esaustiva. \\
Come conseguenza della classe di complessità di appartenenza del problema è chiaro quindi che bisogna comunque risolverlo
in un tempo ragionevole, ossia polinomiale, anche nel caso generale, quindi si ricorre ad algoritmi
di approssimazione adoperando diverse euristiche.
Esistono anche algoritmi esatti non banali, come l'MTP \cite{MTP} o il Bin Completion \cite{BinCompletion} che sono i due
algoritmi esatti più efficienti per il problema trovati fin'ora.


\section{Variazioni e applicazioni}
Ci sono diverse variazioni di questo problema, ognuna con varie applicazioni, tra le più famose troviamo quella
del bin packing a più dimensioni, il packing by weight e il VM packing. Ovviamente ne esistono molte altre ma di seguito 
verranno descritte quelle precedentemente elencate per dare un'idea della ricorrenza di questo problema, nel prossimo 
capitolo si parlerà della variante con bins estendibili nel dettaglio.

\subsection{N-dimensional packing}
Questa variante del problema classico occorre ogni volta che bisogna immagazzinare oggetti in contenitori non in base al
loro volume ma in base alle loro misure in relazione alle dimensioni considerate nel problema. Ad esempio, se si vogliono
riempire dei container con oggetti da spedire e ogni container ha un costo per essere spedito, al fine di minimizzare il costo di spedizione si vuole fare in modo
che vengano usati meno container possibili. Questa è una classica applicazione di 3-dimensional packing, dove appunto si vuole
minimizzare il numero di container utilizzati e assegnare gli oggetti ai container in base alle loro dimensioni, quali larghezza
lunghezza ed altezza.

\subsection{Packing by weight}
In questa variante anziché assegnare gli oggetti in base al loro volume si assegnano in base al loro peso. Si potrebbe fare un
esempio analogo a quello del N-dimensional packing, ovvero se bisogna spedire della merce, e questa merce deve essere trasportata
da dei camion, è risaputo che ogni camion può trasportare merce il cui peso totale è al massimo uguale ad una certa soglia, quindi 
l'obiettivo sarà quello di minimizzare i camion da utilizzare riempendoli con merce il cui peso totale non superi quella di tolleranza
del generico camion.

\subsection{VM packing}
Questo problema prende il nome direttamente dal suo caso di utilizzo maggiore, ovvero il packing delle pagine di memoria
condivise da delle Virtual Machines su uno stesso server.\\
In pratica, in questa variante, gli oggetti facenti parte di determinati insiemi possono condividere il loro spazio quando sono contenuti nello 
stesso bin, quindi ne occupando di meno che quando sono in bin diversi, poiché aumenterebbero lo spazio occupato 
per colpa della loro taglia individuale. L'esempio di applicazione è quello già menzionato, ovvero se ci sono più Virtual Machines
su un server si vuole ridurre la loro occupazione di memoria, quindi si può pensare di condividere lo spazio riguardante le pagine di memoria
che le VMs utilizzano e se più VMs richiedono la stessa pagina questa non avrà bisogno di essere replicata ma solo referenziata,
allora l'obiettivo è quello di minimizzare i blocchi di memoria occupata considerando il fatto che determinate pagine possono
condividere spazio tra loro.

\subsection{Nota}
Si può notare che se il numero di bins è ristretto a uno solo, e gli oggetti oltre ad avere un peso hanno anche un valore, il 
problema di massimizzare il valore degli oggetti che possono essere contenuti dal bin è nient'altro che il problema dello zaino 0-1. \\
In generale tutti i cosiddetti \textit{Knapsack Problems} possono essere visti come casi speciali del Bin Packing.
