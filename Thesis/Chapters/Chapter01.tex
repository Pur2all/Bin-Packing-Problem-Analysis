\section{Il problema}
Il problema del bin packing è così formulato:
\begin{quote}
	Dati degli oggetti di volume differente, questi devono essere allocati in un numero finito di
	contenitori (bin), ciascuno di una determinata capacità fissa, in modo da ridurre al minimo la
	quantità di contenitori utilizzata.
\end{quote}
\noindent
Un altro modo di formulare il problema è attraverso la programmazione lineare intera:
\begin{quote}	
	\begin{equation*}
		\begin{array}{ll@{}ll}
			\text{min}  & \displaystyle\sum\limits_{i=1}^{n} & y_i 					  & \\
			\text{s.t.} & \displaystyle\sum\limits_{j=1}^{n} & w_j x_{ij} \leq c y_i  & \forall i \in \{1, ..., n\}    \\
		                & \displaystyle\sum\limits_{i=1}^{n} & x_{ij} = 1 			  & \forall j \in \{1, ..., n\}    \\
		             	& 								     & y_{i} \in \{0,1\}	  & \forall i \in \{1, ..., n\}    \\
		                & 								     & x_{ij} \in \{0,1\}  	  & \forall i, j \in \{1, ..., n\} \\
		\end{array}
	\end{equation*}
	\begin{equation*}
		\begin{array}{ll@{}ll}
	    	\text{dove} &									& \\
	    				& w_j = \text{peso dell'oggetto } j	\text{, } w_j \in \mathbb{Z}^+ \text{, } w_j \leq c \text{ } \forall j \in \{1, ..., n\}  & \\ \\ 
		    			& c = \text{capacità di ogni bin, } c \in \mathbb{N} & \\ \\
		    			& y_i = 
		    				\begin{cases}
    							1 & \text{se il bin } i \text{ è utilizzato} \\
    							0 & \text{altrimenti}
							\end{cases}					    & \\ \\
						& x_{ij} = 
							\begin{cases}
    							1 & \text{se l'oggetto } j \text{ è assegnato al bin } i \\
    							0 & \text{altrimenti}
							\end{cases}					    & \\
		\end{array}
	\end{equation*}
\end{quote}
\noindent
In modo da rendere più chiara la comprensione viene fornito un esempio:
\begin{quote}
	Sia $ S $ un insieme di cardinalità $ n $ contenente i vari oggetti, indicati da un numero intero positivo, e $ s(i) $
	il volume dei vari oggetti $ \forall i \in S $ e sia $ c $ la capacità di un generico bin. \\
	Consideriamo la seguente istanza:
	\begin{equation*}
		\begin{array}{c}
			S = \{1, ..., 10\} \text{, quindi } |S| = n = 10 \text{, } c = 15 \\
	    \end{array}
	\end{equation*}
	\begin{equation*}
	    \begin{array}{cc}
			s(1) = 15	&	s(6) = 7   \\
			s(2) = 3	&	s(7) = 13  \\
			s(3) = 9	&	s(8) = 1   \\
			s(4) = 5	&	s(9) = 2   \\
			s(5) = 3	&	s(10) = 11 \\
		\end{array}
	\end{equation*}
	Una soluzione ottima per l'istanza è data dai bins:
	\begin{align*}
		B_1 = \{1\} ,\quad B_2 = \{7, 9\} ,\quad B_3 = \{4, 5, 6\} ,\quad B_4 = \{2, 8, 10\} ,\quad B_5 = \{3\}
	\end{align*}
	E quindi si può concludere che il numero minimo di bins da utilizzare per quest'istanza è $ 5 $.
\end{quote}
Ci sono diverse variazioni di questo problema e ognuna ha varie applicazioni, come ad esempio distribuire il carico di
lavoro su dei server, salvare dei file di varie dimensioni in blocchi di memoria di dimensione fissa o questioni di
scheduling dove il tempo d'esecuzione di un generico task è il peso dell'oggetto e i contenitori sono le risorse che li
completano e così via, si possono fare davvero molti esempi di applicazioni, in particolare si può notare che se il numero
di bins è ristretto a uno solo, e gli oggetti oltre ad avere un peso hanno anche un valore, il problema di massimizzare il
valore degli oggetti che possono essere contenuti dal bin è nient'altro che il problema dello zaino. \\ \\
Questo problema è chiaramente di natura combinatorica, e in particolare appartiene all'insieme dei problemi NP-hard, mentre
la versione di decisione del problema fa parte dell'insieme dei problemi NP-complete ed è così definita \cite{CaI_NP}:
\begin{quote}
	Istanza: Insieme finito $ U $ di oggetti, una taglia $ s(u) \in \mathbb{Z}^+ \: \forall u \in U $, una capacità intera
	per i bins $ B $ e un intero positivo $ K $. \\
	Domanda: Esiste una partizione di $ U $ in insiemi disgiunti $ U_1, U_2, ..., U_k $ tale che la somma delle taglie degli
	oggetti in ogni $ U_i $ è $ \leq B $.
\end{quote}
In particolare è un problema strongly NP-completo e questo può essere provato tramite una riduzione polinomiale da un altro
problema strongly NP-completo ovvero il 3-partition. \\
Nonostante ciò comunque è possibile risolvere il problema del bin packing in tempo pseudo-polinomiale per ogni numero di
bins fissato $ \geq 2 $ \cite{BinPackingFixedSize} e in tempo polinomiale per
ogni fissata capacità $ B $ enumerando tutte le possibili $ k^{p(1) + p(2) + ... + p(B)} $ assegnazioni degli oggetti ai
bins e trovare la più conveniente tramite ricerca esaustiva. \\
Come conseguenza della classe di complessità di appartenenza del problema è chiaro quindi che bisogna comunque che esso
venga risolto in un tempo ragionevole, ossia polinomiale, anche nel caso generale, di conseguenza si ricorre ad algoritmi
di approssimazione adoperando diverse euristiche.
Esistono comunque algoritmi esatti, come l'MTP \cite{MTP} o il Bin Completion \cite{BinCompletion}.