Nel problema del Bin Packing, degli oggetti di volume diverso devono essere allocati in un numero finito di contenitori 
(bin), ciascuno di una determinata capacità fissa, in modo da ridurre al minimo il numero di bin utilizzati. Il Bin Packing 
ha numerose applicazioni. Per esempio l'inserimento di file con dimensioni specifiche in blocchi di memoria di dimensioni fisse 
o la registrazione di tutta la musica di un compositore, dove la lunghezza dei pezzi da registrare (in bytes) è la dimensione 
degli oggetti e la capacità del generico bin è la quantità di bytes che può essere memorizzata su un CD audio. Il problema del 
Bin Packing è un noto problema NP-hard, per cui è irrealistico pensare che si possano trovare algoritmi di complessità polinomiale 
per la sua soluzione. In questa tesi, analizziamo e valutiamo sperimentalmente due algoritmi per un'importante variante del Bin 
Packing, in cui è possibile "riempire" ogni bin anche al disopra della sua capacità. L'obiettivo è di minimizzare la somma dei 
valori assoluti delle differenze tra le capacità dei bin e il volume totale del loro contenuto.