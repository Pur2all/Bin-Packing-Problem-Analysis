\section{Il problema}
Nell'ambito dell'informatica, precisamente quando si parla di sistemi distribuiti e concorrenza, nasce il problema
del bilanciamento del carico, ovvero ridurre il carico di lavoro per le risorse distribuendolo tra le altre disponibili, questo 
non è altro che una variante del problema del Bin Packing di cui segue la formulazione.

\subsection{Formulazione}
In questa versione del Bin Packing Problem abbiamo un numero fissato di bins, un numero fissato di oggetti molto maggiore di 
quello dei bins e la capacità di ogni singolo bin può essere ecceduta. Inoltre la somma dei volumi dei vari oggetti è pari alla 
somma delle capacità dei vari bins. L'obiettivo è quello di minimizzare la quantità data dalla somma dei valori assoluti delle 
differenze tra le capacità del bin e la somma del volume dei vari oggetti che essi contengono. \\
Formalmente:
\begin{quote}
	Dati $ m $ bins $ \{B_1, ..., B_m\} $ con capacità $ c $ uguale per tutti e $ n $ oggetti $ \{o_1, ..., o_n\} $ con $ n >> m $ e di volume
	arbitrario (che indicheremo con $ s(o_i) \: \forall i \in \{1, ..., n\} $), tali che $ \displaystyle\sum\limits_{i=1}^{n} s(o_i) = mc $,
	bisogna allocare gli $ n $ oggetti negli $ m $ bins in modo da minimizzare la quantità $ \displaystyle\sum\limits_{i=1}^{m} |c - x_i| $
	dove $ x_i = \displaystyle\sum\limits_{o \in B_i} s(o) \: \forall i \in \{1, ..., m\} $.
\end{quote}
Una possibile formulazione come un problema di programmazione lineare intera è la seguente:
\begin{quote}	
	\begin{equation*}
		\begin{array}{ll@{}ll}
			\text{min}  & \displaystyle\sum\limits_{i=1}^{m} & |c - \displaystyle\sum\limits_{j=1}^{n} x_{ij}w_j|   & \\
			\text{s.t.} & \displaystyle\sum\limits_{i=1}^{n} & x_{ij} = 1 			  								& \forall j \in \{1, ..., n\} \\ 
		                & 								     														& x_{ij} \in \{0,1\}  	  & \forall i, j \in \{1, ..., n\} \\
		\end{array}
	\end{equation*}
	\begin{equation*}
		\begin{array}{ll@{}ll}
	    	\text{dove} &																									   & \\
	    				& w_j = \text{volume dell'oggetto } j	\text{, } w_j \in \mathbb{Z}^+ \: \forall j \in \{1, ..., n\}  & \\ \\
		    			& c = \text{capacità di ogni bin, } c \in \mathbb{N} 												   & \\ \\
						& x_{ij} = 
							\begin{cases}
    							1 & \text{se l'oggetto } j \text{ è assegnato al bin } i \\
    							0 & \text{altrimenti}
							\end{cases}					    																   & \\
		\end{array}
	\end{equation*}
	\begin{equation*}
		\begin{array}{c}
			\text{ed è stata assunta soddisfatta la seguente condizione:} \\
			\displaystyle\sum\limits_{j=1}^{n} w_j = mc					 \\								  
		\end{array}
	\end{equation*}
\end{quote}


\section{Due possibili algoritmi}
Di seguito verranno illustrati due algoritmi di approssimazione che risolvono il problema del Bin Packing con bins estendibili 
dei quali verrà analizzato il comportamento da un punto di vista sperimentale nel prossimo capitolo.

\subsection{Algoritmo con euristica MBO}
Il primo algoritmo utilizza un'euristica nota per i problemi di scheduling multiprocessore, l'euristica Longest Processing Time, ovvero
quella secondo cui si assegnano i job alle risorse in base al loro tempo di esecuzione, precisamente vengono assegnati in ordine non decrescente di 
tempo di esecuzione alla risorsa più "libera" al fine di minimizzare il \textit{makespan}, ovvero il massimo tempo di utilizzo totale di una risorsa tra le varie.
Il problema dello scheduling multiprocessore è molto simile, si potrebbe dire equivalente, a questa variante del Bin Packing. \\
In questo caso, poiché il problema non è di scheduling ma di Bin Packing, potremmo chiamare l'euristica Most Big Object e quindi procedere nel
seguente modo per ottenere una soluzione:
\begin{algorithm}[H]
\begin{algorithmic}[1]

\Function{binPackMBO}{$ B , O , s() $}
    \State $ Ordina \: gli \: oggetti \: in \: modo \: non \: decrescente \: in \: base \: al \: volume $
    \State $ Q = B $

    \For{$ i = 1; \: i \leq |O|; \: i = i + 1 $}
        \State $ \text{bin-più-vuoto} \leftarrow $ \Call{extractMax}{$ Q, s() $}
        \State $ \text{bin-più-vuoto} \leftarrow \text{bin-più-vuoto} \cup \{O[i]\} $
        \State $ Q \leftarrow Q \cup \{\text{bin-più-vuoto}\} $
    \EndFor
    
    \State \Return $ \displaystyle\sum\limits_{b \in B} |b| $
\EndFunction

\end{algorithmic}
\end{algorithm}

\noindent
L'algoritmo prende in input l'insieme di bins $ B $, la lista di oggetti $ O $ e la funzione taglia $ s() $. Dopo aver ordinato gli oggetti in modo
non decrescente, utilizzando il loro volume come metro comparativo, procede con l'assegnazione degli oggetti ai bins in base a quale sia quello più
vuoto al momento dell'assegnazione, semplicemente scorrendo la lista di oggetti da sinistra a destra, quindi dal più voluminoso al meno voluminoso, ed estraendo
dall'insieme $ Q $ il bin di taglia (capacità) maggiore, ovvero quello più vuoto, e modificando quindi implicitamente la capacità di quest'ultimo diminuendola
del peso dell'oggetto assegnatogli, per poi rimetterlo in $ Q $. Infine ritorna il valore a cui si è interessati. 

\subsubsection{Analisi della complessità dell'algoritmo}
L'istruzione al rigo \texttt{2} prende tempo $ O(n \log{}n) $ dove $ n $ è la lunghezza della lista di oggetti, il for delle righe \texttt{4 - 7} viene eseguito $ n $
volte, l'istruzione del rigo \texttt{5}, se l'insieme $ Q $ è un semplice insieme la ricerca deve essere effettuata in modo esaustivo, quindi l'operazione di estrazione del 
massimo ha complessità $ \Theta(m) $ dove $ m $ è la cardinalità di $ Q $ (quindi il numero di bins), le altre istruzioni all'interno del \texttt{for} prendono tempo costante, e 
l'espressione nel \texttt{return} alla fine prende tempo $ \Theta(m) $. Quindi la complessità dell'algoritmo in questo caso è $ O(nm) $ (se $ m > \log{}n $) per via della 
ricerca esaustiva che viene fatta $ n $ volte all'interno del for, ma se $ Q $ fosse uno heap l'operazione di estrazione del massimo prenderebbe tempo $ \Theta(\log{}m) $, 
quella di inserimento in $ Q $ prenderebbe anch'essa tempo $ \Theta(\log{}m) $ ma se ne trae vantaggio perché la complessità del \texttt{for} diventerebbe $ O(n \log{}m) $ che è minore di $ O(nm) $
e quindi la complessità dell'algoritmo diventerebbe $ O(n \log{}n) $, ovvero dominata dall'istruzione al rigo \texttt{2}. L'algoritmo sarà quindi implementato usando uno heap
come struttura per $ Q $.

\subsubsection{Esempio di esecuzione}
Anche se non verrà rispettata la condizione per cui $ n >> m $ al fine di rendere più semplice l'esempio questo non rappresenta una perdita di generalità.
\begin{quote}
	Sia $ O $ una lista di lunghezza $ n $ contenente i vari oggetti, e $ s() $
	la funzione di taglia e sia $ B $ un insieme di cardinalità $ m $ contenente i vari bin ognuno di capacità $ c $. \\
	Consideriamo la seguente istanza:
	\begin{equation*}
		\begin{array}{c}
			O = \{o_1, ..., o_{10}\} \text{, quindi } |O| = n = 10 \text{ e} \\
			B = \{B_1, B_2, B_3\} \text{, quindi } |B| = m = 3 \text{, } c = 18
	    \end{array}
	\end{equation*}
	\begin{equation*}
	    \begin{array}{cc}
			s(o_1) = 4	&	s(o_6) = 7   \\
			s(o_2) = 5	&	s(o_7) = 9  \\
			s(o_3) = 4	&	s(o_8) = 3   \\
			s(o_4) = 5	&	s(o_9) = 3   \\
			s(o_5) = 5	&	s(o_{10}) = 9 \\
		\end{array}
	\end{equation*}
	
	L'esecuzione dell'algoritmo è la seguente:
	\begin{equation*}
		\begin{array}{c}
			O = [o_{10}, o_7, o_6, o_5, o_4, o_2, o_1, o_3, o_8, o_9] \\
			Q = B 
		\end{array}
	\end{equation*}
	\begin{table}[H]
  		\begin{center}
    	\caption{Esecuzione per ogni iterazione}
    		\begin{tabular}{c|c|c|c}
      			\textbf{Iterazione} & \textbf{Volume oggetto} & \textbf{Bin più vuoto} & \textbf{Nuova capacità}\\
      			$ i $ & $ s(O[i]) $ & $ \displaystyle\max_{j \in \{1, ... , m\}} s(B_j) $ & $ s(B_j) - s(O[i]) $ \\
      			\hline
      			1 & 9 & $ B_1, \: s(B_1) = 18 $ & $ s(B_1) = 18 - 9 = 9 $ \\
      			2 & 9 & $ B_2, \: s(B_2) = 18 $ & $ s(B_2) = 18 - 9 = 9 $ \\
      			3 & 7 & $ B_3, \: s(B_3) = 18 $ & $ s(B_3) = 18 - 7 = 11 $ \\
      			4 & 5 & $ B_3, \: s(B_3) = 11 $ & $ s(B_3) = 11 - 5 = 6 $ \\
      			5 & 5 & $ B_2, \: s(B_2) = 9 $ & $ s(B_2) = 9 - 5 = 4 $ \\
      			6 & 5 & $ B_1, \: s(B_1) = 9 $ & $ s(B_1) = 9 - 5 = 4 $ \\
      			7 & 4 & $ B_3, \: s(B_3) = 6 $ & $ s(B_3) = 6 - 4 = 2 $ \\
      			8 & 4 & $ B_2, \: s(B_2) = 4 $ & $ s(B_2) = 4 - 4 = 0 $ \\
      			9 & 3 & $ B_1, \: s(B_1) = 4 $ & $ s(B_1) = 4 - 3 = 1 $ \\
      			10 & 3 & $ B_3, \: s(B_3) = 2 $ & $ s(B_3) = 2 - 3 = -1 $ \\
    		\end{tabular}
  		\end{center}
	\end{table}
	Infine risulterà $ B_1 = \{o_{10}, o_2, o_8\}, B_2 = \{o_7, o_4, o_3\}, B_3 = \{o_6, o_5, o_1, o_9\} $
	quindi $ s(B_1) = 0, s(B_2) = 1, s(B_3) = -1 $ ed il valore tornato dall'algoritmo sarà $ |0| + |1| + |-1| = 2 $.
\end{quote}
\noindent
È da notare che la soluzione ottenuta non è ottima, in quanto assegnando gli oggetti ai bins nel modo seguente si otterrebbe il valore
della funzione obiettivo uguale a $ 0 $:

\begin{center}
	$ B_1 = \{o_{10}, o_7\}, B_2 = \{o_5, o_6, o_8, o_9\}, B_3 = \{o_1, o_2, o_3, o_4\} $.
\end{center}

\noindent
Ci sono comunque dei casi in cui questo algoritmo trova la soluzione ottima sempre, ovvero quando c'è un oggetto $ o_k $ tale che 
$ s(o_k) \geq \displaystyle\frac{\sum_{i=1, i \neq k}^n s(o_i)}{m - 1} $. \\ \\ 
Ad esempio se viene data in input un'istanza con $ m $ bins di capacità $ m $ e $ n = m(m - 1) + 1 $ oggetti e per i primi $ n - 1 $ oggetti 
$ s(o_i) = 1 \: \forall i \in \{1, ..., n - 1\} $ e per l'ultimo oggetto $ s(o_n) = m $ l'algoritmo troverà la soluzione ottima.
Infatti l'algoritmo assegnerebbe al primo bin l'oggetto $ o_n $ e i restanti $ m - 1 $ bins sarebbero riempiti ad ogni iterazione 
con un oggetto di volume unitario $ m $ volte ciascuno, quindi alla fine delle iterazioni si ritroveranno saturati e la soluzione 
fornita dall'algoritmo avrà valore $ 0 $ che inoltre sarà anche quella ottima.

\subsection{Algoritmo con euristica merging sugli oggetti}
Questo secondo algoritmo utilizza un'euristica di merging sugli oggetti che si basa sull'idea di rendere il numero di oggetti e di
bins uguale per poi assegnarne ognuno in modo arbitrario a un bin, per avere che i numeri di oggetti e di bins coincidano fonde insieme gli oggetti
con volume minimo in modo da crearne uno nuovo il cui volume è la somma di quelli che lo compongono.\\
Per avere una soluzione quindi si può procedere nel seguente modo:
\begin{algorithm}[H]
\begin{algorithmic}[1]

\Function{binPackMERGING}{$ B , O , s() $}
    \State $ Q = O $

    \While{$ |B| \neq |Q| $}
        \State $ \text{obj-meno-grande1} \leftarrow $ \Call{extractMin}{$ Q, s() $}
        \State $ \text{obj-meno-grande2} \leftarrow $ \Call{extractMin}{$ Q, s() $}
        \State $ \text{nuovo-obj} \leftarrow \text{obj-meno-grande1} \cup \text{obj-meno-grande2} $
        \State $ Q \leftarrow Q \cup \{\text{nuovo-obj}\} $
    \EndWhile
    
    \For{$ i = 1; \: i \leq |B|; \: i = i + 1 $}
    	\State $ B_i = B_i \cup \{q_i\} $
    \EndFor
    
    \State \Return $ \displaystyle\sum\limits_{b \in B} |b| $
\EndFunction

\end{algorithmic}
\end{algorithm}

\noindent
L'algoritmo prende in input l'insieme di bins $ B $, la lista di oggetti $ O $ e la funzione taglia $ s() $. Dopo aver assegnato gli oggetti della lista $ O $
all'insieme $ Q $ procede con l'operazione di merging estraendo dall'insieme $ Q $ i due oggetti con volume minimo per poi unirli creando un nuovo oggetto, che
avrà come volume la somma dei volumi degli altri due, e inserirlo in $ Q $, questo fin quando non risulta $ n = m $ ($ n $ numero di oggetti e $ m $ numero di bins). 
Una volta terminata l'operazione di merging si procede con l'assegnazione degli oggetti ai bin in maniera arbitraria ed infine viene ritornato il valore a cui si è interessati.

\subsubsection{Analisi della complessità dell'algoritmo}
L'istruzione al rigo \texttt{2} prende tempo costante, il while delle righe \texttt{3 - 7} viene eseguito 
$ n - m $ volte, le istruzioni delle righe \texttt{4} e \texttt{5}, se l'insieme $ Q $ è un semplice insieme la ricerca deve essere effettuata in modo esaustivo, quindi l'operazione di 
estrazione del minimo ha complessità $ \Theta(n) $, le altre istruzioni all'interno del \texttt{while} prendono tempo costante, il \texttt{for} delle righe \texttt{8 - 9} viene eseguito 
$ m $ volte e l'istruzione al suo interno prende tempo costante, infine l'espressione del \texttt{return} prende tempo $ \Theta(m) $. Quindi la complessità dell'algoritmo
in questo caso è $ O(n(n - m)) $ per via della ricerca esaustiva che viene fatta $ n - m $ volte all'interno del \texttt{while}, ma se $ Q $ fosse uno heap l'operazione di 
estrazione del minimo prenderebbe tempo $ \Theta(\log{}n) $, quella di inserimento in $ Q $ prenderebbe anch'essa tempo $ \Theta(\log{}n) $ ma se ne trae vantaggio
perché la complessità del \texttt{while} diventerebbe $ O((n - m) \log{}n) $ che è minore di $ O(n(n - m)) $ e quindi la complessità dell'algoritmo diventerebbe proprio questa,
ovvero dominata dalla ripetizione delle istruzioni all'interno del \texttt{while}. L'algoritmo sarà quindi implementato usando uno heap come struttura per $ Q $.

\subsubsection{Esempio di esecuzione}
Anche se non verrà rispettata la condizione per cui $ n >> m $ al fine di rendere più semplice l'esempio questo non rappresenta una perdita di generalità.
\begin{quote}
	Sia $ O $ una lista di lunghezza $ n $ contenente i vari oggetti, e $ s() $
	la funzione di taglia e sia $ B $ un insieme di cardinalità $ m $ contenente i vari bin ognuno di capacità $ c $. \\
	Consideriamo la seguente istanza:
	\begin{equation*}
		\begin{array}{c}
			O = \{o_1, ..., o_{10}\} \text{, quindi } |O| = n = 10 \text{ e} \\
			B = \{B_1, B_2, B_3\} \text{, quindi } |B| = m = 3 \text{, } c = 18
	    \end{array}
	\end{equation*}
	\begin{equation*}
	    \begin{array}{cc}
			s(o_1) = 4	&	s(o_6) = 7   \\
			s(o_2) = 5	&	s(o_7) = 9  \\
			s(o_3) = 4	&	s(o_8) = 3   \\
			s(o_4) = 5	&	s(o_9) = 3   \\
			s(o_5) = 5	&	s(o_{10}) = 9 \\
		\end{array}
	\end{equation*}
	
	L'esecuzione dell'algoritmo è la seguente:
	\begin{equation*}
		\begin{array}{c}
			Q = O 
		\end{array}
	\end{equation*}
	\begin{table}[H]
  		\begin{center}
    	\caption{Esecuzione per ogni iterazione del while}
    		\begin{tabular}{c|c|c|c}
      			\textbf{Iterazione} & \textbf{Oggetto}		  					 & \textbf{Oggetto}											& \textbf{Nuovo oggetto} \\
      								& \textbf{meno grande 1}  					 & \textbf{meno grande 2} 									& $ s(nuovo \: obj) $ \\
      			\hline
      			1 					& $ o_9, s(o_9) = 3 $  					  	 & $ o_8, s(o_8) = 3 $ 										& $ o_{9 \cup 8}, 6 $ \\
      			2 					& $ o_3, s(o_3) = 4 $  					  	 & $ o_1, s(o_1) = 4 $ 										& $ o_{3 \cup 1}, 8 $ \\
      			3 					& $ o_2, s(o_2) = 5 $  					  	 & $ o_4, s(o_4) = 5 $ 										& $ o_{2 \cup 4}, 10 $ \\
      			4 					& $ o_5, s(o_5) = 5 $  					  	 & $ o_{9 \cup 8}, s(o_{9 \cup 8}) = 6 $ 					& $ o_{5 \cup 9 \cup 8}, 11 $ \\
      			5 					& $ o_6, s(o_6) = 7 $  					  	 & $ o_{3 \cup 1}, s(o_{3 \cup 1}) = 8 $ 					& $ o_{6 \cup 3 \cup 1}, 15 $ \\
      			6 					& $ o_7, s(o_7) = 9 $  					  	 & $ o_{10}, s(o_{10}) = 9 $ 								& $ o_{7 \cup 10}, 18 $\\
      			7 					& $ o_{2 \cup 4}, s(o_{2 \cup 4}) = 10 $  	 & $ o_{5 \cup 9 \cup 8}, s(o_{5 \cup 9 \cup 8}) = 11 $  	& $ o_{2 \cup 4 \cup 5 \cup 9 \cup 8}, 21 $ \\
    		\end{tabular}
  		\end{center}
	\end{table}
	Infine risulterà $ B_1 = \{o_{6 \cup 3 \cup 1}\}, B_2 = \{o_{7 \cup 10}\}, B_3 = \{o_{2 \cup 4 \cup 5 \cup 9 \cup 8}\} $
	quindi $ s(B_1) = 3, s(B_2) = 0, s(B_3) = -3 $ ed il valore tornato dall'algoritmo sarà $ |3| + |0| + |-3| = 6 $.
\end{quote}
\noindent
È da notare che la soluzione ottenuta non è ottima, in quanto unendo gli oggetti e assegnandoli ai bins nel modo seguente si otterrebbe il valore
della funzione obiettivo uguale a $ 0 $:

\begin{center}
	$ B_1 = \{o_{10 \cup 7}\}, B_2 = \{o_{5 \cup 6 \cup 8 \cup 9}\}, B_3 = \{o_{1 \cup 2 \cup 3 \cup 4}\} $.
\end{center}

\noindent
Anche per questo algoritmo ci sono casi in cui trova la soluzione ottima, una condizione sufficiente affinché questa cosa si realizzi
è quella di avere $ m $ bins e $ n $ oggetti tale che $ n = 2^k, k \geq 2 $ e $ m = 2^h, 1 \leq h < k $ e che tutti gli oggetti abbiano taglia uguale,
in particolare $ s(o_i) = cm/n \: \forall i \in \{1, ..., n\} $.
Ad esempio se $ m = 2 $, $ c = 16 $ e $ n = 8 $ allora ogni oggetto ha taglia $ 2 $, l'algoritmo assegnerà gli oggetti in modo
ottimale, avendo valore della funzione obiettivo pari a $ 0 $.
