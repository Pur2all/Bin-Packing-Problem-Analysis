\phantomsection
\addcontentsline{toc}{section}{Argomento trattato}
\section*{Argomento trattato}
All'interno di questa tesi si è trattato di algoritmi che risolvono il problema del Bin Packing con 
bins estendibili. Il problema è quello di mettere un numero grande di oggetti all'interno di 
un numero molto più piccolo di contenitori che possono essere estesi se la loro capacità viene superata, con lo scopo 
di minimizzare lo spazio vuoto e in eccesso totale dei contenitori. \\
Per questo problema non esiste un algoritmo polinomiale che lo risolve all'ottimo, infatti fa parte della
classe di problemi NP-hard, di conseguenza bisogna approssimare la soluzione nel miglior modo
possibile tramite algoritmi che fanno uso di euristiche.

\phantomsection
\addcontentsline{toc}{section}{Obiettivo}
\section*{Obiettivo}
L'obiettivo della tesi è quello di analizzare sperimentalmente due algoritmi di approssimazione 
per il problema del Bin Packing con bins estendibili, per trarre in conclusione qual è il migliore tra i due 
in base alle prestazioni e risultati per il problema in esame. \\
Tali algoritmi si basano su due euristiche diverse, il primo opera inserendo ogni volta l'oggetto di volume 
maggiore all'interno del contenitore più vuoto, mentre il secondo unisce gli oggetti di volume minore 
fin quando il loro numero non corrisponde a quello dei contenitori e li inserisce in modo arbitrario in essi.

\phantomsection
\addcontentsline{toc}{section}{Organizzazione della tesi}
\section*{Organizzazione della tesi}
La tesi è suddivisa in 3 capitoli, all'interno del primo capitolo si formalizza il problema del Bin Packing
classico, si parla della sua complessità e si introducono alcune sue variazioni; all'interno del secondo si formalizza
il problema del Bin Packing con bins estendibili e vengono descritti i due algoritmi che saranno analizzati; infine, all'
interno del terzo capitolo si discute dell'approccio utilizzato alla sperimentazione, gli strumenti utilizzati e vengono forniti i dati
da cui sono state fatte le considerazioni finali, anch'esse presenti all'interno di questo capitolo.
