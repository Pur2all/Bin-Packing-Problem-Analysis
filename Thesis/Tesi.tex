\begin{filecontents*}[overwrite, noheader]{\jobname.xmpdata}
	\Title{Un'analisi sperimentale di algoritmi per Bin Packing}
	\Author{Francesco Migliaro}
	\Language{it-IT}
	\Subject{Nel problema del Bin Packing, degli oggetti di volume diverso devono essere allocati in un numero finito di contenitori 
			(bin), ciascuno di una determinata capacità fissa, in modo da ridurre al minimo il numero di bin utilizzati. Il Bin Packing 
			ha numerose applicazioni. Per esempio l'inserimento di file con dimensioni specifiche in blocchi di memoria di dimensioni fisse 
			o la registrazione di tutta la musica di un compositore, dove la lunghezza dei pezzi da registrare (in bytes) è la dimensione 
			degli oggetti e la capacità del generico bin è la quantità di bytes che può essere memorizzata su un CD audio. Il problema del 
			Bin Packing è un noto problema NP-hard, per cui è irrealistico pensare che si possano trovare algoritmi di complessità polinomiale 
			per la sua soluzione. In questa tesi, analizziamo e valutiamo sperimentalmente due algoritmi per un'importante variante del Bin 
			Packing, in cui è possibile "riempire" ogni bin anche al disopra della sua capacità. L'obiettivo è di minimizzare la somma dei 
			valori assoluti delle differenze tra le capacità dei bin e il volume totale del loro contenuto.​}
	\Keywords{Bin Packing\sep Algoritmi di approssimazione\sep Algoritmi}
\end{filecontents*}

\documentclass[a4paper, titlepage]{report}

\usepackage[T1]{fontenc}
\usepackage[utf8]{inputenc}
\usepackage[italian]{babel}
\usepackage{amsmath, amssymb}
\usepackage{float}
\usepackage[backend=biber, sorting=none]{biblatex}
\usepackage{algorithm}
\usepackage[noend]{algpseudocode}
\usepackage{listings}
\usepackage[cache=false]{minted}
\usepackage[toc, page]{appendix}
\usepackage{xcolor}
\usepackage[autostyle, italian=guillemets]{csquotes}
\usepackage{tabularx}
\usepackage{graphicx}
\usepackage{colorprofiles}
\usepackage[evenboxes, classica]{topfront}
\usepackage{titletoc, tocloft}
\usepackage[a-2b]{pdfx}
\usepackage[depth=4]{bookmark}

\hypersetup{
    pdfpagemode={UseOutlines},
    bookmarksopen,
    bookmarksnumbered,
    pdfstartview={FitH},
    colorlinks,
    linkcolor={black}, %COLORE DEI RIFERIMENTI AL TESTO
    citecolor={blue}, %COLORE DEI RIFERIMENTI ALLE CITAZIONI
    urlcolor={blue} %COLORI DEGLI URL
}

\setcounter{tocdepth}{4}
\setcounter{secnumdepth}{3}

\cftsetindents{paragraph}{130pt}{4em}

\usemintedstyle{my_monokai_light}

\restylefloat{table}

\renewcommand\appendixpagename{Appendice}
\renewcommand\appendixtocname{Appendice}

\graphicspath{ {./Images/Experiments/} }

\renewcommand{\Candidato}{Candidato}
\makeatletter
      \def\fr@ntespizio{%
            \begingroup\par
                  \oddsidemargin=\dimexpr(\oddsidemargin+\evensidemargin)/2\relax
                  \evensidemargin \oddsidemargin
            \null
            \setcounter{page}{1}%
            \normalfont
            \ifclassica
                  \boolfalse{topTPTlogos}
                  \thispagestyle{classica}
                  \ifcsvoid{@ateneo}{\def\@ateneo{Manca il nome dell'ateneo}
                  }{}
            \else
                  \thispagestyle{titlepage}
            \fi
            \ifcsvoid{@ateneo}{}{\booltrue{AteneoInHead}}
            \ifcsvoid{@ateneo}{%
                  \ifbool{topTPTlogos}
                  {}{\booltrue{AteneoInHead}\def\@ateneo{Manca il nome dell'ateneo}}%
            }{%
                  \booltrue{AteneoInHead}%
            }
            
            \unless\iftopTPTlogos
                  {\centering \printloghi\par}
            \fi
            \vspace*{\stretch{0.1}}

            \ifbool{AteneoInHead}{}{%
                  {{\centering\LARGE \@ateneo\par}}
            }
            \ifcsvoid{@nomeat}{}
                  {\ifbool{topTPTlogos}{\vspace*{\dimexpr \headsep+2.5ex}}{\vspace*{-3ex}}%
                  {\centering\@nomeat\par}\vfill}

                  \begin{center}
                  {\rmfamily\mdseries
                  \ifdottorato
                        \large \@phdschool\par\medskip
                  \else
                        \ifcsvoid{@faculty}{}{%
                        \LARGE\ifx\@facnumber\empty\else\@facnumber\space\fi
                        \@faculty\unskip\xspace\@facname\par\medskip
                        }
                  \fi
                  }%
                  \ifcsvoid{@corso}{}{{\large
                        \ifdottorato
                              \@PhDname\unskip\xspace
                                    \@corso\ifx\@ciclo\empty\else~--~\@ciclo\fi
                        \else
                              \@laureaname\unskip\xspace\@corso
                        \fi
                        \par}}
                  \end{center}
                  \vspace*{\stretch{0.5}}
                  \begin{center}
                        \LARGE
                        \ifdottorato
                              \@dissertazione%
                        \else
                              \iftriennale
                                    \@monografia%
                              \else
                                    \@TesiDiLaurea%
                              \fi
                        \fi
                        \unless\ifx\empty\@materia
                              \  \@InName \  \@materia
                        \fi
                  \end{center}
                  \vspace*{\stretch{0}}
                  \begin{center}
                        {\huge\bfseries \baselineskip=0.95em plus 1pt
                              \@titolo \par}
                  \end{center}
                  \unless\ifx\@subtitle\empty
                        \begin{center}%
                        	\large\textrm{\@subtitle}\par
                        \end{center}%
                  \fi
                  \vspace*{\stretch{1}}
                  \ifclassica
                        \ifnum\value{tomo}>\z@
                              \par\bigskip
                              \noindent\makebox[\textwidth]{%
                              \large\textbf{%
                              \ifcase\c@tomo%
                                    \or \PrimoTomo%
                                    \or \SecondoTomo%
                                    \or \TerzoTomo%[]
                                    \or \QuartoTomo%
                                    \else
                                          \PackageWarning{toptesi}{%
                                          Counter tomo equals
                                                \the\c@tomo\MessageBreak
                                          We never considered a thesis might get
                                          divided in more than four volumes}%
                                    \fi}}%
                        \fi
                        \vspace{1em}
                  \fi
                  \par
                  \iftriennale
                        \let\@nomerelatore\empty
                  \else
                        \ifdottorato
                              \edef\@nomerelatore{\@PhDdirector}%
                        \else
                              \ifcsvoid{@principaladviser}{}{%
                                    \def\@nomerelatore{\Relatore}}
                              \unless\ifclassica
                                    \ifcsvoid{@secondadviser}{}{%
                                          \def\@nomerelatore{\Relatori}}%
                              \fi
                        \fi
                  \fi
                  \ifdottorato
                        \let\@nomecandidato\empty
                  \else
                  \iflanguage{italian}{%
                        \iffemminile
                              \def\@nomecandidato{\Candidata}%
                        \else
                              \def\@nomecandidato{\Candidato}%
                        \fi
                        \ifcsvoid{@secondauthor}{}{%
                              \iffemminile
                                    \def\@nomecandidato{\Candidate}%
                              \else
                                    \def\@nomecandidato{\Candidati}%
                              \fi}
                  }{}%
            \fi
                  \iftriennale
                        \begin{center}%
                              \large\mdseries\textsc{\@author}
                        \end{center}%
                  \else
                  \def\BoxRelatori{%
                        \begin{tabular}[t]{l}%
                        \hbox{\ifclassica\else\large\fi
                              \textbf{\protect\@nomerelatore}}\\[.6ex]
                     
                              \hbox{\large\tabular{@{}l@{}}\@principaladviser\endtabular}%
                        \ifx\@secondadviser\empty \else
                              \ifclassica
                                    \ifx\@thirdadviser\empty
                                          \ifx\@secondadviser\empty\else
                                                \\[1.5ex]\textbf{\Correlatore:}%
                                    \fi
                              \else
                                    \\[1.5ex]\textbf{\Correlatori:}%
                              \fi
                        \fi
                        \\[.6ex]\hbox{{\large\textrm{\protect\@secondadviser}}}%
                        \fi
                        \ifx\@thirdadviser\empty \else
                              \\[.6ex] \hbox{{\large\textrm{\protect\@thirdadviser}}}%
                        \fi
                        \end{tabular}%
                  }%
                  \def\print@secondocandidato{\\\relax
                              \hbox{\large\tabular{@{}l@{}}\@secondauthor\endtabular}}
                  \def\print@terzocandidato{\\\relax
                              \hbox{\large\tabular{@{}l@{}}\@thirdauthor\endtabular}}
                  \def\BoxCandidati{%
                  \begin{tabular}[t]{l}%
                        \hfill\hbox{\unless\ifclassica\large\fi
                              \textbf{\protect\@nomecandidato}}\\[.6ex]
                              \hbox{\large\tabular{@{}l@{}}\@author\endtabular}%
                              \ifcsvoid{@secondauthor}{}{\print@secondocandidato}%
                              \ifcsvoid{@thirdauthor}{}{\print@terzocandidato}%
                  \end{tabular}%
                  }%
                  \ifdottorato
                        \begin{center}\large
                              \textbf{\@author}\\[3em]
                              {\normalsize
                              \begin {tabular*}{\hsize}{@{\extracolsep{\fill}}ccc}
                                    \ifcsvoid{@tutore}{}{\textbf{\Tutore}}
                              &\relax&
                                    \ifcsvoid{@principaladviser}{}{\textbf{\@nomerelatore}}
                              \\
                                    \ifcsvoid{@tutore}{}{\@tutore}
                              &\relax&
                                    \ifcsvoid{@principaladviser}{}{\@principaladviser}
                              \end{tabular*}
                              }%
                        \end{center}
                  \else
                  \unless\ifclassica
                        \unless\ifevenboxes
                        \begin{flushleft}%
                              \BoxRelatori
                        \end{flushleft}\par\vspace*{-1.5\baselineskip}
                        \begin{flushright}%
                              \BoxCandidati
                        \end{flushright}%
                        \else
                              \makebox[\textwidth]{\BoxRelatori\hfill\BoxCandidati}
                        \fi
                        \else
                              \noindent
                              \makebox[\textwidth]{%
                              \BoxRelatori\hfill\BoxCandidati}\par
                        \fi
                  \fi
            \fi
            \vspace*{\stretch{0.2}}
            \ifcsvoid{@tutoreaziendale}{}{%
            \vfill\vfill
            {\centering \textbf{\@tutoreaziendalename}\\[.6ex]
            \@tutoreaziendale\par}}
                  \par\clearpage
                  \ifcsvoid{@retrofrontespizio}{}%
                  {\null\vfill\thispagestyle{empty}\@retrofrontespizio\par\clearpage}%
            \endgroup}
\makeatother

\addbibresource{bibliography.bib}

\begin{document}

\begin{frontespizio}
	\ateneo{Università degli Studi di Salerno}
	\logosede{Images/logo_standard.png}

	\StrutturaDidattica{Dipartimento di}
	\struttura{Informatica}
	
	\corsodilaurea{Informatica}
	
	\titolo{Un'analisi sperimentale di algoritmi per Bin Packing}
	\Materia{Informatica}
	\relatore{Ch.mo Prof. Ugo Vaccaro}
	\candidato{\hfill Francesco Migliaro\\ \hfill Matricola 0512105109}
	\annoaccademico{2019-2020}
\end{frontespizio}

\pagenumbering{roman}

\pdfbookmark[section]{\contentsname}{toc}
\tableofcontents

\cleardoublepage
\phantomsection
\addcontentsline{toc}{chapter}{Abstract}
\chapter*{Abstract}
Nel problema del Bin Packing, degli oggetti di volume diverso devono essere allocati in un numero finito di contenitori 
(bin), ciascuno di una determinata capacità fissa, in modo da ridurre al minimo il numero di bin utilizzati. Il Bin Packing 
ha numerose applicazioni. Per esempio l'inserimento di file con dimensioni specifiche in blocchi di memoria di dimensioni fisse 
o la registrazione di tutta la musica di un compositore, dove la lunghezza dei pezzi da registrare (in bytes) è la dimensione 
degli oggetti e la capacità del generico bin è la quantità di bytes che può essere memorizzata su un CD audio. Il problema del 
Bin Packing è un noto problema NP-hard, per cui è irrealistico pensare che si possano trovare algoritmi di complessità polinomiale 
per la sua soluzione. In questa tesi, analizziamo e valutiamo sperimentalmente due algoritmi per un'importante variante del Bin 
Packing, in cui è possibile "riempire" ogni bin anche al disopra della sua capacità. L'obiettivo è di minimizzare la somma dei 
valori assoluti delle differenze tra le capacità dei bin e il volume totale del loro contenuto.

\cleardoublepage
\phantomsection
\addcontentsline{toc}{chapter}{\listfigurename}
\listoffigures

\cleardoublepage
\phantomsection
\addcontentsline{toc}{chapter}{\listtablename}
\listoftables

\cleardoublepage
\pagenumbering{arabic}
\phantomsection
\addcontentsline{toc}{chapter}{Introduzione}
\chapter*{Introduzione}
\phantomsection
\addcontentsline{toc}{section}{Argomento trattato}
\section*{Argomento trattato}
All'interno di questa tesi si è trattato di algoritmi che risolvono il problema del Bin Packing con 
bins estendibili. Il problema è quello di mettere un numero grande di oggetti all'interno di 
un numero molto più piccolo di contenitori che possono essere estesi se la loro capacità viene superata, con lo scopo 
di minimizzare lo spazio vuoto e in eccesso totale dei contenitori. \\
Per questo problema non esiste un algoritmo polinomiale che lo risolve all'ottimo, infatti fa parte della
classe di problemi NP-hard, di conseguenza bisogna approssimare la soluzione nel miglior modo
possibile tramite algoritmi che fanno uso di opportune euristiche.

\phantomsection
\addcontentsline{toc}{section}{Obiettivo}
\section*{Obiettivo}
L'obiettivo della tesi è quello di analizzare sperimentalmente due algoritmi di approssimazione 
per il problema del Bin Packing con bins estendibili, in modo da vedere qual è il migliore tra i due. \\
Tali algoritmi si basano su due euristiche diverse, il primo opera inserendo ogni volta l'oggetto di volume 
maggiore all'interno del contenitore più vuoto, mentre il secondo unisce gli oggetti di volume minore 
fin quando il loro numero non corrisponde a quello dei contenitori e li inserisce in modo arbitrario in essi.

\phantomsection
\addcontentsline{toc}{section}{Organizzazione della tesi}
\section*{Organizzazione della tesi}
La tesi è suddivisa in 3 capitoli, all'interno del primo capitolo si formalizza il problema del Bin Packing
classico, si parla della sua complessità e si introducono alcune sue variazioni; all'interno del secondo si formalizza
il problema del Bin Packing con bins estendibili e vengono descritti i due algoritmi che saranno analizzati; infine, all'
interno del terzo capitolo si discute dell'approccio utilizzato alla sperimentazione, gli strumenti utilizzati e vengono forniti i dati
da cui sono state fatte le considerazioni finali, anch'esse presenti all'interno di questo capitolo.

\phantomsection
\addcontentsline{toc}{section}{Conclusioni da trarre}
\section*{Conclusioni da trarre}
Le conclusioni che si vogliono trarre dal lavoro svolto riguardano la scelta di usare un determinato
algoritmo rispetto ad un altro in base alle sue prestazioni e risultati per il problema in esame.

\chapter{Bin Packing Problem}
\section{Il problema}
Spesso ci si può ritrovare a dover risolvere problemi in cui si devono disporre degli oggetti in contenitori al fine
di utilizzarne il minor numero possibile, ad esempio se si deve prendere un aereo bisogna organizzare tutte le proprie cose in delle
valigie da portare con sé, valigie che ovviamente avranno un costo di trasporto, di conseguenza si è interessati ad
utilizzare il minimo numero di valigie al fine di spendere il meno possibile per i bagagli. Questa è proprio una
classica applicazione del problema del bin packing, un problema che è più frequente di quanto si immagini.

\subsection{Formulazione}
Il problema del bin packing è così formulato:
\begin{quote}
	Dati degli oggetti di volume differente, questi devono essere allocati in un numero finito di
	contenitori (bin), ciascuno di una determinata capacità fissa, in modo da ridurre al minimo la
	quantità di contenitori utilizzata.
\end{quote}
\noindent
Un altro modo di formulare il problema è attraverso la programmazione lineare intera:
\begin{quote}	
	\begin{equation*}
		\begin{array}{ll@{}ll}
			\text{min}  & \displaystyle\sum\limits_{i=1}^{n} & y_i 					  & \\
			\text{s.t.} & \displaystyle\sum\limits_{j=1}^{n} & w_j x_{ij} \leq c y_i  & \forall i \in \{1, ..., n\}    \\
		                & \displaystyle\sum\limits_{i=1}^{n} & x_{ij} = 1 			  & \forall j \in \{1, ..., n\}    \\
		             	& 								     & y_{i} \in \{0,1\}	  & \forall i \in \{1, ..., n\}    \\
		                & 								     & x_{ij} \in \{0,1\}  	  & \forall i, j \in \{1, ..., n\} \\
		\end{array}
	\end{equation*}
	\begin{equation*}
		\begin{array}{ll@{}ll}
	    	\text{dove} &									& \\
	    				& w_j = \text{peso dell'oggetto } j	\text{, } w_j \in \mathbb{Z}^+ \text{, } w_j \leq c \text{ } \forall j \in \{1, ..., n\}  & \\ \\ 
		    			& c = \text{capacità di ogni bin, } c \in \mathbb{N} & \\ \\
		    			& y_i = 
		    				\begin{cases}
    							1 & \text{se il bin } i \text{ è utilizzato} \\
    							0 & \text{altrimenti}
							\end{cases}					    & \\ \\
						& x_{ij} = 
							\begin{cases}
    							1 & \text{se l'oggetto } j \text{ è assegnato al bin } i \\
    							0 & \text{altrimenti}
							\end{cases}					    & \\
		\end{array}
	\end{equation*}
\end{quote}
\noindent
In modo da rendere più chiara la comprensione viene fornito un esempio:
\begin{quote}
	Sia $ S $ un insieme di cardinalità $ n $ contenente i vari oggetti, indicati da un numero intero positivo, e $ s(i) $
	il volume dei vari oggetti $ \forall i \in S $ e sia $ c $ la capacità di un generico bin. \\
	Consideriamo la seguente istanza:
	\begin{equation*}
		\begin{array}{c}
			S = \{1, ..., 10\} \text{, quindi } |S| = n = 10 \text{, } c = 15 \\
	    \end{array}
	\end{equation*}
	\begin{equation*}
	    \begin{array}{cc}
			s(1) = 15	&	s(6) = 7   \\
			s(2) = 3	&	s(7) = 13  \\
			s(3) = 9	&	s(8) = 1   \\
			s(4) = 5	&	s(9) = 2   \\
			s(5) = 3	&	s(10) = 11 \\
		\end{array}
	\end{equation*}
	Una soluzione ottima per l'istanza è data dai bins:
	\begin{align*}
		B_1 = \{1\} ,\quad B_2 = \{7, 9\} ,\quad B_3 = \{4, 5, 6\} ,\quad B_4 = \{2, 8, 10\} ,\quad B_5 = \{3\}
	\end{align*}
	E quindi si può concludere che il numero minimo di bins da utilizzare per quest'istanza è $ 5 $.
\end{quote}

\subsection{Difficoltà}
Questo problema è chiaramente di natura combinatoria, e in particolare appartiene all'insieme dei problemi NP-hard, mentre
la versione di decisione del problema fa parte dell'insieme dei problemi NP-complete ed è così definita \cite{CaI_NP}:
\begin{quote}
	Istanza: Insieme finito $ U $ di oggetti, una taglia $ s(u) \in \mathbb{Z}^+ \: \forall u \in U $, una capacità intera
	per i bins $ B $ e un intero positivo $ K $. \\
	Domanda: Esiste una partizione di $ U $ in insiemi disgiunti $ U_1, U_2, ..., U_k $ tale che la somma delle taglie degli
	oggetti in ogni $ U_i $ è $ \leq B $.
\end{quote}
Nello specifico è un problema strongly NP-complete e questo può essere provato tramite una riduzione polinomiale da un altro
problema strongly NP-complete, ovvero il 3-partition. \\
Nonostante ciò comunque è possibile risolvere il problema del bin packing in tempo pseudo-polinomiale per ogni numero di
bins fissato $ \geq 2 $ \cite{BinPackingFixedSize}, e in tempo polinomiale per
ogni fissata capacità $ B $ enumerando tutte le possibili $ k^{p(1) + p(2) + ... + p(B)} $ assegnazioni degli oggetti ai
bins e trovare la più conveniente tramite ricerca esaustiva. \\
Come conseguenza della classe di complessità di appartenenza del problema è chiaro quindi che bisogna comunque risolverlo
in un tempo ragionevole, ossia polinomiale, anche nel caso generale, quindi si ricorre ad algoritmi
di approssimazione adoperando diverse euristiche.
Esistono anche algoritmi esatti non banali, come l'MTP \cite{MTP} o il Bin Completion \cite{BinCompletion} che sono i due
algoritmi esatti più efficienti per il problema trovati fin'ora.


\section{Variazioni e applicazioni}
Ci sono diverse variazioni di questo problema, ognuna con varie applicazioni, tra le più famose troviamo quella
del bin packing a più dimensioni, il packing by weight e il VM packing. Ovviamente ne esistono molte altre ma di seguito 
verranno descritte quelle precedentemente elencate per dare un'idea della ricorrenza di questo problema, nel prossimo 
capitolo si parlerà della variante con bins estendibili nel dettaglio.

\subsection{N-dimensional packing}
Questa variante del problema classico occorre ogni volta che bisogna immagazzinare oggetti in contenitori non in base al
loro volume ma in base alle loro misure in relazione alle dimensioni considerate nel problema. Ad esempio, se si vogliono
riempire dei container con oggetti da spedire e ogni container ha un costo per essere spedito, al fine di minimizzare il costo di spedizione si vuole fare in modo
che vengano usati meno container possibili. Questa è una classica applicazione di 3-dimensional packing, dove appunto si vuole
minimizzare il numero di container utilizzati e assegnare gli oggetti ai container in base alle loro dimensioni, quali larghezza
lunghezza ed altezza.

\subsection{Packing by weight}
In questa variante anziché assegnare gli oggetti in base al loro volume si assegnano in base al loro peso. Si potrebbe fare un
esempio analogo a quello del N-dimensional packing, ovvero se bisogna spedire della merce, e questa merce deve essere trasportata
da dei camion, è risaputo che ogni camion può trasportare merce il cui peso totale è al massimo uguale ad una certa soglia, quindi 
l'obiettivo sarà quello di minimizzare i camion da utilizzare riempendoli con merce il cui peso totale non superi quella di tolleranza
del generico camion.

\subsection{VM packing}
Questo problema prende il nome direttamente dal suo caso di utilizzo maggiore, ovvero il packing delle pagine di memoria
condivise da delle Virtual Machines su uno stesso server.\\
In pratica, in questa variante, gli oggetti facenti parte di determinati insiemi possono condividere il loro spazio quando sono contenuti nello 
stesso bin, quindi ne occupando di meno che quando sono in bin diversi, poiché aumenterebbero lo spazio occupato 
per colpa della loro taglia individuale. L'esempio di applicazione è quello già menzionato, ovvero se ci sono più Virtual Machines
su un server si vuole ridurre la loro occupazione di memoria, quindi si può pensare di condividere lo spazio riguardante le pagine di memoria
che le VMs utilizzano e se più VMs richiedono la stessa pagina questa non avrà bisogno di essere replicata ma solo referenziata,
allora l'obiettivo è quello di minimizzare i blocchi di memoria occupata considerando il fatto che determinate pagine possono
condividere spazio tra loro.

\subsection{Nota}
Si può notare che se il numero di bins è ristretto a uno solo, e gli oggetti oltre ad avere un peso hanno anche un valore, il 
problema di massimizzare il valore degli oggetti che possono essere contenuti dal bin è nient'altro che il problema dello zaino 0-1. \\
In generale tutti i cosiddetti \textit{Knapsack Problems} possono essere visti come casi speciali del Bin Packing.


\chapter{Bin Packing Problem con bins estendibili}
\section{Il problema}
Nell'ambito dell'informatica, precisamente quando si parla di sistemi distribuiti e concorrenza, nasce il problema
del bilanciamento del carico, ovvero ridurre il carico di lavoro per le risorse distribuendolo tra le altre disponibili, questo 
non è altro che una variante del problema del bin packing di cui segue la formulazione.

\subsection{Formulazione}
Questa versione del Bin Packing Problem è una variante del problema precedentemente illustrato nel primo capitolo, nella quale abbiamo un numero
fissato di bins, un numero fissato di oggetti molto maggiore di quello dei bins e la capacità di ogni singolo bin può essere ecceduta.
Inoltre la somma dei volumi dei vari oggetti è pari alla somma delle capacità dei vari bins.
L'obiettivo è quello di minimizzare la quantità data dalla somma dei valori assoluti delle differenze tra la capacità del bin e la somma del
volume dei vari oggetti che esso contiene. \\
Formalmente:
\begin{quote}
	Dati $ m $ bins $ \{B_1, ..., B_m\} $ con capacità $ c $ uguale per tutti e $ n $ oggetti $ \{o_1, ..., o_n\} $ con $ n >> m $ e di volume
	arbitrario (che indicheremo con $ s(o_i) \: \forall i \in \{1, ..., n\} $), tali che $ \displaystyle\sum\limits_{i=1}^{n} s(o_i) = mc $,
	bisogna allocare gli $ n $ oggetti negli $ m $ bins in modo da minimizzare la quantità $ \displaystyle\sum\limits_{i=1}^{m} |c - x_i| $
	dove $ x_i = \displaystyle\sum\limits_{o \in B_i} s(o) \: \forall i \in \{1, ..., m\} $.
\end{quote}
Una possibile formulazione come un problema di programmazione lineare intera è la seguente:
\begin{quote}	
	\begin{equation*}
		\begin{array}{ll@{}ll}
			\text{min}  & \displaystyle\sum\limits_{i=1}^{m} & |c - \displaystyle\sum\limits_{j=1}^{n} x_{ij}w_j|   & \\
			\text{s.t.} & \displaystyle\sum\limits_{i=1}^{n} & x_{ij} = 1 			  								& \forall j \in \{1, ..., n\} \\ 
		                & 								     														& x_{ij} \in \{0,1\}  	  & \forall i, j \in \{1, ..., n\} \\
		\end{array}
	\end{equation*}
	\begin{equation*}
		\begin{array}{ll@{}ll}
	    	\text{dove} &																									   & \\
	    				& w_j = \text{volume dell'oggetto } j	\text{, } w_j \in \mathbb{Z}^+ \: \forall j \in \{1, ..., n\}  & \\ \\
		    			& c = \text{capacità di ogni bin, } c \in \mathbb{N} 												   & \\ \\
						& x_{ij} = 
							\begin{cases}
    							1 & \text{se l'oggetto } j \text{ è assegnato al bin } i \\
    							0 & \text{altrimenti}
							\end{cases}					    																   & \\
		\end{array}
	\end{equation*}
	\begin{equation*}
		\begin{array}{c}
			\text{ed è stata assunta soddisfatta la seguente condizione:} \\
			\displaystyle\sum\limits_{j=1}^{n} w_j = mc					 \\								  
		\end{array}
	\end{equation*}
\end{quote}


\section{Due possibili algoritmi}
Di seguito verranno illustrati due algoritmi di approssimazione che risolvono il problema del bin packing con bins estendibili 
dei quali verrà analizzato il comportamento da un punto di vista sperimentale nel prossimo capitolo.

\subsection{Algoritmo con euristica LPT}
Il primo algoritmo utilizza un'euristica nota per i problemi di scheduling multiprocessore, l'euristica Longest Processing Time, ovvero
quella secondo cui si assegnano i job alle risorse in base al loro tempo di esecuzione, precisamente vengono assegnati in ordine non decrescente di 
tempo di esecuzione alla risorsa più "libera" al fine di minimizzare il \textit{makespan}, ovvero il massimo tempo di utilizzo totale di una risorsa tra le varie.
Il problema dello scheduling multiprocessore è molto simile, si potrebbe dire equivalente, a questa variante del bin packing. \\
In questo caso, poiché il problema non è di scheduling ma di bin packing, potremmo chiamare l'euristica Most Big Object e quindi procedere nel
seguente modo per ottenere una soluzione:
\begin{algorithm}[H]
\begin{algorithmic}[1]

\Function{binPackLPT}{$ B , O , s() $}
    \State $ Ordina \: gli \: oggetti \: in \: modo \: non \: decrescente $
    \State $ Q = B $

    \For{$ i = 1; \: i \leq |O|; \: i = i + 1 $}
        \State $ \text{bin-più-vuoto} \leftarrow $ \Call{extractMax}{$ Q, s() $}
        \State $ \text{bin-più-vuoto} \leftarrow \text{bin-più-vuoto} \cup \{O[i]\} $
        \State $ Q \leftarrow Q \cup \{\text{bin-più-vuoto}\} $
    \EndFor
    
    \State \Return $ \displaystyle\sum\limits_{b \in B} |b| $
\EndFunction

\end{algorithmic}
\end{algorithm}

\noindent
L'algoritmo prende in input l'insieme di bins $ B $, la lista di oggetti $ O $ e la funzione taglia $ s() $. Dopo aver ordinato gli oggetti in modo
non decrescente, utilizzando il loro volume come metro comparativo, procede con l'assegnazione degli oggetti ai bins in base a quale sia quello più
vuoto al momento dell'assegnazione, semplicemente scorrendo la lista di oggetti da sinistra a destra, quindi dal più voluminoso al meno voluminoso, ed estraendo
dall'insieme $ Q $ il bin di taglia (capacità) maggiore, ovvero quello più vuoto, e modificando quindi implicitamente la capacità di quest'ultimo diminuendola
del peso dell'oggetto da assegnare, per poi rimetterlo in $ Q $. Infine ritorna il valore a cui si è interessati. 

\paragraph{Analisi della complessità dell'algoritmo}\mbox{}\\
L'istruzione al rigo \texttt{2} prende tempo $ O(n \log{}n) $ dove $ n $ è la lunghezza della lista di oggetti, il for delle righe \texttt{4 - 7} viene eseguito $ n $
volte, l'istruzione del rigo \texttt{5}, se l'insieme $ Q $ è un semplice insieme la ricerca deve essere effettuata in modo esaustivo, quindi l'operazione di estrazione del 
massimo ha complessità $ \Theta(m) $ dove $ m $ è la cardinalità di $ Q $ (quindi il numero di bins), le altre istruzioni all'interno del \texttt{for} prendono tempo costante, e 
l'espressione nel \texttt{return} alla fine prende tempo $ \Theta(m) $. Quindi la complessità dell'algoritmo in questo caso è $ O(nm) $ (se $ m \geq \log{}n $) per via della 
ricerca esaustiva che viene fatta $ n $ volte all'interno del for, ma se $ Q $ fosse uno heap l'operazione di estrazione del massimo prenderebbe tempo $ \Theta(\log{}m) $, 
quella di inserimento in $ Q $ prenderebbe anch'essa tempo $ \Theta(\log{}m) $ ma se ne trae vantaggio perché la complessità del \texttt{for} diventerebbe $ O(n \log{}m) $ che è minore di $ O(nm) $
e quindi la complessità dell'algoritmo diventerebbe $ O(n \log{}n) $, ovvero dominata dall'istruzione al rigo \texttt{2}.

\paragraph{Esempio di esecuzione}\mbox{}\\
Anche se non verrà rispettata la condizione per cui $ n >> m $ al fine di rendere più semplice l'esempio questo non rappresenta una perdita di generalità.
\begin{quote}
	Sia $ O $ una lista di lunghezza $ n $ contenente i vari oggetti, e $ s() $
	la funzione di taglia e sia $ B $ un insieme di cardinalità $ m $ contenente i vari bin ognuno di capacità $ c $. \\
	Consideriamo la seguente istanza:
	\begin{equation*}
		\begin{array}{c}
			O = \{o_1, ..., o_{10}\} \text{, quindi } |O| = n = 10 \text{ e} \\
			B = \{B_1, B_2, B_3\} \text{, quindi } |B| = m = 3 \text{, } c = 18
	    \end{array}
	\end{equation*}
	\begin{equation*}
	    \begin{array}{cc}
			s(o_1) = 4	&	s(o_6) = 7   \\
			s(o_2) = 5	&	s(o_7) = 9  \\
			s(o_3) = 4	&	s(o_8) = 3   \\
			s(o_4) = 5	&	s(o_9) = 3   \\
			s(o_5) = 5	&	s(o_{10}) = 9 \\
		\end{array}
	\end{equation*}
	
	L'esecuzione dell'algoritmo è la seguente:
	\begin{equation*}
		\begin{array}{c}
			O = [o_{10}, o_7, o_6, o_5, o_4, o_2, o_1, o_3, o_8, o_9] \\
			Q = B 
		\end{array}
	\end{equation*}
	\begin{table}[H]
  		\begin{center}
    	\caption{Esecuzione per ogni iterazione}
    		\begin{tabular}{c|c|c|c}
      			\textbf{Iterazione} & \textbf{Volume oggetto} & \textbf{Bin più vuoto} & \textbf{Nuova capacità}\\
      			$ i $ & $ s(O[i]) $ & $ \displaystyle\max_{j \in \{1, ... , m\}} s(B_j) $ & $ s(B_j) - s(O[i]) $ \\
      			\hline
      			1 & 9 & $ B_1, \: s(B_1) = 18 $ & $ s(B_1) = 18 - 9 = 9 $ \\
      			2 & 9 & $ B_2, \: s(B_2) = 18 $ & $ s(B_2) = 18 - 9 = 9 $ \\
      			3 & 7 & $ B_3, \: s(B_3) = 18 $ & $ s(B_3) = 18 - 7 = 11 $ \\
      			4 & 5 & $ B_3, \: s(B_3) = 11 $ & $ s(B_3) = 11 - 5 = 6 $ \\
      			5 & 5 & $ B_2, \: s(B_2) = 9 $ & $ s(B_2) = 9 - 5 = 4 $ \\
      			6 & 5 & $ B_1, \: s(B_1) = 9 $ & $ s(B_1) = 9 - 5 = 4 $ \\
      			7 & 4 & $ B_3, \: s(B_3) = 6 $ & $ s(B_3) = 6 - 4 = 2 $ \\
      			8 & 4 & $ B_2, \: s(B_2) = 4 $ & $ s(B_2) = 4 - 4 = 0 $ \\
      			9 & 3 & $ B_1, \: s(B_1) = 4 $ & $ s(B_1) = 4 - 3 = 1 $ \\
      			10 & 3 & $ B_3, \: s(B_3) = 2 $ & $ s(B_3) = 2 - 3 = -1 $ \\
    		\end{tabular}
  		\end{center}
	\end{table}
	Infine risulterà $ B_1 = \{o_{10}, o_2, o_8\}, B_2 = \{o_7, o_4, o_3\}, B_3 = \{o_6, o_5, o_1, o_9\} $
	quindi $ s(B_1) = 0, s(B_2) = 1, s(B_3) = -1 $ ed il valore tornato dall'algoritmo sarà $ |0| + |1| + |-1| = 2 $.
\end{quote}
\noindent
È da notare che la soluzione ottenuta non è ottima, in quanto assegnando gli oggetti ai bins nel modo seguente si otterrebbe il valore
della funzione obiettivo uguale a $ 0 $:

\begin{center}
	$ B_1 = \{o_{10}, o_7\}, B_2 = \{o_5, o_6, o_8, o_9\}, B_3 = \{o_1, o_2, o_3, o_4\} $.
\end{center}

\noindent
Ci sono comunque dei casi in cui questo algoritmo trova la soluzione ottima sempre, ovvero quando c'è un oggetto $ o_k $ tale che 
$ s(o_k) \geq \displaystyle\frac{\sum_{i=1, i \neq k}^n s(o_i)}{m - 1} $. \\ \\ 
Ad esempio se viene data in input un'istanza con $ m $ bins di capacità $ m $ e $ n = m(m - 1) + 1 $ oggetti e per i primi $ n - 1 $ oggetti 
$ s(o_i) = 1 \: \forall i \in \{1, ..., n - 1\} $ e per l'ultimo oggetto $ s(o_n) = m $ l'algoritmo troverà la soluzione ottima.
Infatti l'algoritmo assegnerebbe al primo bin l'oggetto $ o_n $ e i restanti $ m - 1 $ bins sarebbero riempiti ad ogni iterazione 
con un oggetto di volume unitario $ m $ volte ciascuno, quindi alla fine delle iterazioni si ritroveranno saturati e la soluzione 
fornita dall'algoritmo avrà valore $ 0 $ che inoltre sarà anche quella ottima.

\subsection{Algoritmo con euristica merging sugli oggetti}
Questo secondo algoritmo utilizza un'euristica di merging sugli oggetti che si basa sull'idea di rendere il numero di oggetti e di
bins uguale per poi assegnarne ognuno in modo arbitrario a un bin, per avere che i numeri di oggetti e di bins coincidano fonde insieme gli oggetti
con volume minore in modo da crearne uno nuovo il cui volume è la somma di quelli che lo compongono.\\
Per avere una soluzione quindi si può procedere nel seguente modo:
\begin{algorithm}[H]
\begin{algorithmic}[1]

\Function{binPackMERGING}{$ B , O , s() $}
    \State $ Q = O $

    \While{$ |B| \neq |Q| $}
        \State $ \text{obj-meno-grande1} \leftarrow $ \Call{extractMin}{$ Q, s() $}
        \State $ \text{obj-meno-grande2} \leftarrow $ \Call{extractMin}{$ Q, s() $}
        \State $ \text{nuovo-obj} \leftarrow \text{obj-meno-grande1} \cup \text{obj-meno-grande2} $
        \State $ Q \leftarrow Q \cup \{\text{nuovo-obj}\} $
    \EndWhile
    
    \For{$ i = 1; \: i \leq |B|; \: i = i + 1 $}
    	\State $ B_i = B_i \cup \{q_i\} $
    \EndFor
    
    \State \Return $ \displaystyle\sum\limits_{b \in B} |b| $
\EndFunction

\end{algorithmic}
\end{algorithm}

\noindent
L'algoritmo prende in input l'insieme di bins $ B $, la lista di oggetti $ O $ e la funzione taglia $ s() $. Dopo aver assegnato gli oggetti della lista $ O $
all'insieme $ Q $ procede con l'operazione di merging estraendo dall'insieme $ Q $ i due oggetti con volume minore per poi unirli creando un nuovo oggetto, che
avrà come volume la somma dei volumi degli altri due, e inserirlo in $ Q $, questo fin quando non risulta $ n = m $ ($ n $ numero di oggetti e $ m $ numero di bins). 
Una volta terminata l'operazione di merging si procede con l'assegnazione degli oggetti ai bin in maniera arbitraria ed infine viene ritornato il valore a cui si è interessati.

\paragraph{Analisi della complessità dell'algoritmo}\mbox{}\\
L'istruzione al rigo \texttt{2} prende tempo costante, il while delle righe \texttt{3 - 7} viene eseguito 
$ n - m $ volte, le istruzioni delle righe \texttt{4} e \texttt{5}, se l'insieme $ Q $ è un semplice insieme la ricerca deve essere effettuata in modo esaustivo, quindi l'operazione di 
estrazione del minimo ha complessità $ \Theta(n) $, le altre istruzioni all'interno del \texttt{while} prendono tempo costante, il \texttt{for} delle righe \texttt{8 - 9} viene eseguito 
$ m $ volte e l'istruzione al suo interno prende tempo costante, infine l'espressione del \texttt{return} prende tempo $ \Theta(m) $. Quindi la complessità dell'algoritmo
in questo caso è $ O(n(n - m)) $ per via della ricerca esaustiva che viene fatta $ n - m $ volte all'interno del \texttt{while}, ma se $ Q $ fosse uno heap l'operazione di 
estrazione del minimo prenderebbe tempo $ \Theta(\log{}n) $, quella di inserimento in $ Q $ prenderebbe anch'essa tempo $ \Theta(\log{}n) $ ma se ne trae vantaggio
perché la complessità del \texttt{while} diventerebbe $ O((n - m) \log{}n) $ che è minore di $ O(n(n - m)) $ e quindi la complessità dell'algoritmo diventerebbe proprio questa,
ovvero dominata dalla ripetizione delle istruzioni all'interno del \texttt{while}.

\paragraph{Esempio di esecuzione}\mbox{}\\
Anche se non verrà rispettata la condizione per cui $ n >> m $ al fine di rendere più semplice l'esempio questo non rappresenta una perdita di generalità.
\begin{quote}
	Sia $ O $ una lista di lunghezza $ n $ contenente i vari oggetti, e $ s() $
	la funzione di taglia e sia $ B $ un insieme di cardinalità $ m $ contenente i vari bin ognuno di capacità $ c $. \\
	Consideriamo la seguente istanza:
	\begin{equation*}
		\begin{array}{c}
			O = \{o_1, ..., o_{10}\} \text{, quindi } |O| = n = 10 \text{ e} \\
			B = \{B_1, B_2, B_3\} \text{, quindi } |B| = m = 3 \text{, } c = 18
	    \end{array}
	\end{equation*}
	\begin{equation*}
	    \begin{array}{cc}
			s(o_1) = 4	&	s(o_6) = 7   \\
			s(o_2) = 5	&	s(o_7) = 9  \\
			s(o_3) = 4	&	s(o_8) = 3   \\
			s(o_4) = 5	&	s(o_9) = 3   \\
			s(o_5) = 5	&	s(o_{10}) = 9 \\
		\end{array}
	\end{equation*}
	
	L'esecuzione dell'algoritmo è la seguente:
	\begin{equation*}
		\begin{array}{c}
			Q = O 
		\end{array}
	\end{equation*}
	\begin{table}[H]
  		\begin{center}
    	\caption{Esecuzione per ogni iterazione del while}
    		\begin{tabular}{c|c|c|c}
      			\textbf{Iterazione} & \textbf{Oggetto}		  					 & \textbf{Oggetto}											& \textbf{Nuovo oggetto} \\
      								& \textbf{meno grande 1}  					 & \textbf{meno grande 2} 									& $ s(nuovo \: obj) $ \\
      			\hline
      			1 					& $ o_9, s(o_9) = 3 $  					  	 & $ o_8, s(o_8) = 3 $ 										& $ o_{9 \cup 8}, 6 $ \\
      			2 					& $ o_3, s(o_3) = 4 $  					  	 & $ o_1, s(o_1) = 4 $ 										& $ o_{3 \cup 1}, 8 $ \\
      			3 					& $ o_2, s(o_2) = 5 $  					  	 & $ o_4, s(o_4) = 5 $ 										& $ o_{2 \cup 4}, 10 $ \\
      			4 					& $ o_5, s(o_5) = 5 $  					  	 & $ o_{9 \cup 8}, s(o_{9 \cup 8}) = 6 $ 					& $ o_{5 \cup 9 \cup 8}, 11 $ \\
      			5 					& $ o_6, s(o_6) = 7 $  					  	 & $ o_{3 \cup 1}, s(o_{3 \cup 1}) = 8 $ 					& $ o_{6 \cup 3 \cup 1}, 15 $ \\
      			6 					& $ o_7, s(o_7) = 9 $  					  	 & $ o_{10}, s(o_{10}) = 9 $ 								& $ o_{7 \cup 10}, 18 $\\
      			7 					& $ o_{2 \cup 4}, s(o_{2 \cup 4}) = 10 $  	 & $ o_{5 \cup 9 \cup 8}, s(o_{5 \cup 9 \cup 8}) = 11 $  	& $ o_{2 \cup 4 \cup 5 \cup 9 \cup 8}, 21 $ \\
    		\end{tabular}
  		\end{center}
	\end{table}
	Infine risulterà $ B_1 = \{o_{6 \cup 3 \cup 1}\}, B_2 = \{o_{7 \cup 10}\}, B_3 = \{o_{2 \cup 4 \cup 5 \cup 9 \cup 8}\} $
	quindi $ s(B_1) = 3, s(B_2) = 0, s(B_3) = -3 $ ed il valore tornato dall'algoritmo sarà $ |3| + |0| + |-3| = 6 $.
\end{quote}
\noindent
È da notare che la soluzione ottenuta non è ottima, in quanto unendo gli oggetti e assegnandoli ai bins nel modo seguente si otterrebbe il valore
della funzione obiettivo uguale a $ 0 $:

\begin{center}
	$ B_1 = \{o_{10 \cup 7}\}, B_2 = \{o_{5 \cup 6 \cup 8 \cup 9}\}, B_3 = \{o_{1 \cup 2 \cup 3 \cup 4}\} $.
\end{center}

\noindent
Anche per questo algoritmo ci sono casi in cui trova la soluzione ottima, una condizione sufficiente affinché questa cosa si realizzi
è quella di avere $ m $ bins e $ n $ oggetti tale che $ n = 2^k, k \geq 2 $ e $ m = 2^h, 1 \leq h < k $ e che tutti gli oggetti abbiano taglia uguale,
in particolare $ s(o_i) = c/n \: \forall i \in \{1, ..., n\} $.
Ad esempio se $ m = 2 $, $ c = 16 $ e $ n = 8 $ allora ogni oggetto ha taglia $ 2 $, l'algoritmo assegnerà gli oggetti in modo
ottimale, avendo valore della funzione obiettivo pari a $ 0 $.


\chapter{Risultati sperimentali}
\section{Il metodo}
Per confrontare i due algoritmi si è proceduto in modo sperimentale, implementando un programma che producesse
un'istanza per il problema per poi essere eseguita su entrambi gli algoritmi e produrre un grafico al fine di 
analizzare i risultati dati. 

\subsection{Generazione degli oggetti e dei bins}
Come prima cosa, dopo aver fissato il numero $ n $ di oggetti e il numero $ m $ di bins, è stato necessario generare le rispettive taglie
in base a qualche criterio in modo che rispettassero le condizioni del problema, ovvero che la somma delle taglie degli oggetti 
deve essere uguale alla somma delle taglie dei bins.

\subsubsection{Una prima soluzione}
La prima soluzione sviluppata per fare in modo che gli oggetti e i bins generati rispettassero le condizioni del problema
è stata quella di scegliere la taglia degli $ m $ bins uguale a $ 1/m $ in modo che la loro somma sia pari a $ 1 $, mentre
per gli oggetti si è proceduto nel seguente modo:
\begin{quote}
	- Si generano $ n $ numeri $ x_i \: \forall i \in \{1, ..., n\} $ in modo che ogni numero sia preso con probabilità 
	uniforme in un intervallo $ [1, M] $. \\
	- Ogni oggetto $ o_i $ avrà taglia pari a $ \displaystyle\frac{x_i}{\sum_{j=1}^n x_j} \: \forall i \in \{1, ..., n\} $.
\end{quote}
Ovvero si è fatto in modo che la somma degli oggetti fosse normalizzata ad $ 1 $.

\paragraph{I problemi di questa soluzione}\mbox{}\\
Nonostante questo criterio di generazione sembri essere buono ai fini degli esperimenti esso comporta dei problemi una volta
realizzato su calcolatore, in quanto così facendo le taglie degli oggetti assumono valori molto piccoli, valori che non vengono
sentiti nella somma tra le varie taglie, e di conseguenza falsano i risultati. Non è possibile fare un confronto tra i due algoritmi
in quanto non effettuano le operazioni nello stesso ordine, un numero potrebbe essere sommato prima o dopo ad un altro in base all'
algoritmo eseguito, e quindi potrebbe amplificare o meno errori di calcolo, inoltre i numeri generati con questo metodo sono
anche soggetti ad arrotondamenti e troncamenti vari al crescere di $ n $ e quindi si perde precisione.


\subsubsection{Una seconda soluzione}
Visti i problemi della prima soluzione illustrata si è cercato di ridurre gli errori dovuti a numeri troppo piccoli adoperando
una tecnica diversa per la generazione degli oggetti ma simile alla prima, ovvero anziché normalizzare a $ 1 $ la somma dei bins 
e quella degli oggetti si è preferito dare una capacità fissa ai bins uguale per tutti e generare le taglie degli oggetti nel
seguente modo:
\begin{quote}
	- Si generano $ n $ numeri $ x_i \: \forall i \in \{1, ..., n\} $ in modo che ogni numero sia preso con probabilità 
	uniforme in un intervallo $ [1, M] $. \\
	- Ogni oggetto $ o_i $ avrà taglia pari a $ \displaystyle\frac{x_i}{\sum_{j=1}^n x_j}cm \: \forall i \in \{1, ..., n\} $. 
\end{quote}
Ovvero si è fatto in modo che la somma degli oggetti fosse normalizzata a $ cm $ dove $ c $ è la capacità fissata
di ogni bin e $ m $ il numero di bins.

\paragraph{I problemi di questa soluzione}\mbox{}\\
Nonostante in questo caso si lavori con numeri più grandi anche questa soluzione come la precedente, anche se di meno, comporta problemi relativi
alla somma effettuata col calcolatore e di conseguenza anche questa tecnica falsa i risultati e non è possibile confrontare
i risultati degli algoritmi.


\subsubsection{Una terza soluzione}
Al fine di eliminare totalmente errori dovuti a calcoli effettuati con numeri non sentiti nella somma su calcolatore 
si è cercato di trovare un modo di generare le taglie degli oggetti per far sì che fossero solo interi, ovvero, una volta fissati 
$ n $ ed $ m $, sia $ sumobj $ la somma delle taglie degli oggetti:
\begin{quote}
	- Si generano $ n $ numeri $ o_i \: \forall i \in \{1, ..., n\} $ in modo che ogni numero sia preso con probabilità
	uniforme in un intervallo $ [1, M] $.\\
	- Se $ sumobj $  non è un multiplo di $ m $ allora si calcola il resto della divisione tra $ sumobj $ ed $ m $ e lo 
	si aggiunge alla taglia di un oggetto scelto a caso, se invece è un multiplo si esegue il prossimo passo direttamente.\\
	- La taglia dei bins diventa $ sumobj/m $.
\end{quote}
In questo modo si lavora solo con numeri interi e gli errori dovuti a numeri non sentiti nella somma con altri sono totalmente
eliminati, di conseguenza i dati sono corretti ed è possibile confrontare gli algoritmi.

\paragraph{I problemi di questa soluzione}\mbox{}\\
Anche se sono stati eliminati del tutto gli errori sui calcoli sorge un altro problema, ovvero quello dell'overflow, infatti se le taglie
degli oggetti sono troppo grandi, o comunque al crescere di $ n $, la somma di questi potrebbe provocare un'overflow, ma questi sono problemi
presenti anche nelle due soluzioni precedenti al crescere di $ n $, anzi le due precedenti potrebbero portare anche ad underflow, ma è trascurabile per
la riuscita dell'esperimento in quanto non è una cosa che può essere risolta in generale.


\subsubsection{La strategia utilizzata}
Si è scelto di utilizzare l'ultima soluzione descritta per generare gli oggetti e i bins in quanto non presenta nessun problema ai fini
dell'esperimento e produce dati corretti, per cui è possibile confrontare i risultati degli algoritmi.


\subsection{Raccolta dati}


\subsection{Analisi grafica}



\section{Implementazione}

\subsection{Linguaggio e librerie utilizzate}


\subsection{Codice}



\section{Dati sperimentali e conclusioni}

\subsection{Dati}


\subsection{Conclusioni}

\printbibliography[heading=bibintoc]

\begin{appendices}
	\section*{Codice}
		\label{sec:Codice}
		\addcontentsline{toc}{section}{Codice}
		\inputminted[linenos, breaklines, tabsize=2]{python3}{../SourceCode/Bin-Packing-Problem-Analysis.py}
\end{appendices}


\end{document}